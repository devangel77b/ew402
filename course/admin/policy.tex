\documentclass[10pt,courier]{navymemo}

\author{Dennis Evangelista}
\title{EW404 Course Policy}
\navysubj{Course policy for EW404 (Robotics and Controls Engineering Design, Construction, and Test)}
\navyfiling{1531}
%\navyserial{16-0001}
\date{\today}
%\navymarking{UNCLASSIFIED}

\usepackage{designature}
\usepackage{siunitx}
\usepackage[colorlinks=true,urlcolor=blue]{hyperref}
%\usepackage{gitinfo2}

\begin{document}
\makedateblock{}

\MEMORANDUM{}

\begin{navyletterheader}
\navyfrom{Assistant Professor D. Evangelista}
\navyto{EW404 advisees}
\navyskip{}%

\navysubjline{}
\navyskip{}%
\navyref{refa}{ACDEANINST 1531.58 (Administration of Academic Programs)}
\navyref{refb}{WEAPS\&SYSENGRINST 5400.7H (Teaching Methods \& Practices)}
\navyref{refc}{USNANINST 1531.53 (Policies Concerning Graded Academic Work)}
%\navyskip{}%
%\navyencl{encl1}{Syllabus, ES200 (Introduction to Systems Engineering), Fall 2016}
\end{navyletterheader}

\section{Requirement}
Per references (a) and (b), this memorandum sets forth my course policy for the Spring 2019 session of EW404 (Robotics and Control Engineering Design, Construction, and Test).  This course acts as a \textbf{capstone} to your undergraduate education in Robotics and Controls Engineering here at the United States Naval Academy.  It provides you with the opportunity to use most of the fundamental knowledge you have acquired in the preceding courses, culminating in a design project, which you are carrying out from concept through construction and test.    This information supplements the basic guidance provided in references~(\ref{refa}) through (\ref{refc}). Successful completion of this course is \textbf{required in order to graduate}.

\section{Course Objectives and Learning Outcomes}  As the capstone experience for the Systems Engineering major, this course will draw upon the entirety of your engineering education to date.  At the completion of this course, you will be able to:
\subsection{} Demonstrate new skills attained in support of the capstone project;
\subsection{} Scope and plan an engineering project of intermediate size, based on engineering insight gained during the execution of the capstone;
\subsection{} Demonstrate successful application of skills taught throughout the program, including troubleshooting, testing, evaluation, team building, decision making, and conflict resolution;
\subsection{} Apply the fundamentals of cost, schedule, and project management learned in ES401 to a novel project;
\subsection{} Demonstrate and analyze and engineering project at its completion, including evluation of both the success of the project and the quality of the initial plan and design;
\subsection{} Demonstrate skills associated with execution of a project test plan, and articulate the need for, and critical elements of, a good test plan;
\subsection{} Construct a complete and professional technical report and presentation that capture the process and results of the capstone effort, including sufficient data and detail to enable review and follow-on work (potentially by others).

\newpage
\section{Expectations}
\subsection{} At a minimum, we will meet once per week. Input from all group members is required; it is considered equivalent to a lecture period for a traditional course. 
\subsection{} Each member of the group is expected to put in at least 10 hours per week of work on the project for this course.
\subsection{} When a technical stumbling block arises, do not wait until the next weekly meeting time to discuss it; arrange EI or send me an email for advice.
\subsection{} If we meet in your work space for a demo, you will be completely set up prior to my arrival (e.g. computer booted, logged on, code opened, parts connected...)
\subsection{} Major deliverables (e.g. report, presentation, Capstone Day poster) will be routed to me, as well as any outside and cross-project reviewers, for comment in draft form at least one week prior to the due date, so that you have time to incorporate my comments on the drafts into the final versions.  Concurrences shall be obtained where appropriate. 
\subsection{} At the end of the course, you will schedule a day to transfer all necessary engineering documentation / knowledge.  This may include: burning code and data to a CD or uploading it to a repository; preparation of system diagrams, wiring diagrams, drawings, etc; providing files used in production (g-code, STL, circuit board layouts, etc.); providing instructions on how to run the demo; turnover of spare materials and supplies and/or equipment return; preparation of a design notebook archiving the entire design history.
\subsection{} I highly recommend doing a test run of your capstone day presentation, for example, in the Biomechanics Seminar series or other venues as appropriate.  

\section{Weekly meetings}  Meetings can be very useful, but are also very costly in terms of person-hours.  Make the world a better place, hold good meetings.  Do not waste\footnote{Gather ye rosebuds while ye may.}  your meetings! Use them to create buy-in/consensus and to drive towards decisions and actions.  \textbf{The evening prior} to our weekly meeting, provide an agenda consisting of:
\subsection{} Weekly status covering technical status, schedule status (revised/updated Gantt chart or scrum board), cost status, and hours logged (for accountability purposes). 
\subsection{} Biggest current challenge and/or technical question you want to discuss.
\subsection{} For any design decisions / course of actions you want to resolve at the meeting, provide your recommendations along with supporting information.
\subsection{} Written meeting minutes shall be kept documenting who was present, what was discussed, and any decisions or commitments made.  You are also highly encouraged to maintain a design notebook. 

\section{Demonstrations} 
\subsection{} Before 6-week grades are due, an interim demonstration or review shall be held to assess my perception of your effort.
\subsection{} During the eighth week of class, we will decide what will be your 12-week demonstration milestone. Aim high, but be realistic.
\subsection{} 12-week demonstration day will be our last scheduled meeting before the day 12-week grades are due.
\subsection{} Plan to provide Capstone Day deliverables ahead of time, and to practice the presentation at a suitable venue, seminar or otherwise. 
\subsection{} Demonstrations shall be ready to go at the \emph{beginning} of the meeting.

\section{Grading}
Interim grades will be based on your performance at weekly meetings, my perception of your effort, and demonstrations as agreed upon.  

In a real engineering project, you will be judged based on on-time delivery of something that meets the requirements, at or under budget, along with complete documentation. You will provide a 12-week demonstration, as well as required reporting in the form of a final poster, presentation, and report. Your final grade will also depend heavily on your performance at weekly meetings and my perception of your effort.

The following general guidelines apply: 
\begin{center}
\begin{tabular}{lp{5in}}
%\SI{30}{\percent} & 12-week demonstration \\
%\SI{10}{\percent} & Performance at weekly meetings \\
%\SI{15}{\percent} & My perception of your effort\\
%\SI{10}{\percent} & Final poster \\
%\SI{10}{\percent} & Final presentation \\
%\SI{25}{\percent} & Final report \\
A & On time, at or under budget, and meeting performance specs \\
B & Behind, but with a path to completion identified \\
C & Critically behind \\
D & Deficient in performance, execution of tasks, planning, or application of engineering principles \\
F & Let's not go here. \\
\end{tabular}
\end{center}

\noclosing{}\\
%\signspace{}
\noindent\hspace*{4in}\includesignature{}
\signature{D Evangelista}
\sendertitle{Assistant Professor}

\noindent\hspace*{4in}{235 Maury Hall}\\
\hspace*{4in}{(410) 293-6132}\\
\hspace*{4in}{\href{mailto:evangeli@usna.edu}{evangeli@usna.edu}}

\copyto{}
File\\
Course coordinator


% record note
\navyrecordnote
\thispagestyle{empty}

\navyrecordnotedistribution{%
Evangelista\\%
Devries\\
Jaramillo}%
%\navyrecordnoteconcurrences{%
%\navyrecordnoteconcurrence{08K}
%\navyrecordnoteconcurrence{08I}}

\navyrecordnotesubjline

\section{}  This memo provides course policies for Evangelista's Spring 2019 EW404 advisees: Biomechanics (Nemani, Hall, Edwards, and Martin) and Toth and Descour (ERCH). 

\section{} The 2018 policy was updated by changing the department name and course number. New in 2019: students are highly encouraged to present before Capstone Day in a suitable venue such as Biomechanics Seminar (Evangelista and Jaramillo lab groups) or as appropriate for their project. 

\section{}  The policies here are adapted from those provided by CDR Severson for ES404.  Evangelista added suggestions to archive code and production files to a repository, system diagrams, turnover of spare materials, supplies and equipment; and to prepare a design notebook, binder or file folder archiving the entire design history.  Evangelista also directs that written meeting minutes shall be kept and meeting agendas will be provided the evening before. A line requesting concurrences as required is added.

%\section{} File information: \gitMark
\end{document}


